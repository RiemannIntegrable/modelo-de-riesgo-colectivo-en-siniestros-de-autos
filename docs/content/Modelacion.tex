\section{Modelación}

En vista del análisis exploratorio, se decidió modelar la pérdida agregada $S$ como la suma de las pérdidas por cobertura, dado que las diferencias en la cantidad de reportes y el valor de los siniestros por cobertura son notables, y mezclarlos en un único análisis sería poco riguroso desde el punto de vista actuarial.

\subsection{Marco teórico del modelo}

Se define $C = \{PPD, PPH, PTH, RC\}$ como el conjunto de coberturas consideradas en el análisis. Para cada cobertura $c \in C$, se establecen las siguientes variables aleatorias:

\begin{itemize}
    \item $X^{(c)}$: Severidad de un siniestro por la cobertura $c$
    \item $\{X^{(c)}_k\}_{k \in \mathbb{N}}$: Muestreo aleatorio de un conjunto de datos que se distribuye como $X^{(c)}$
    \item $N^{(c)}$: Número de siniestros por unidad de tiempo por la cobertura $c$
    \item $S^{(c)} = \sum_{k=1}^{N^{(c)}} X^{(c)}_k$: Pérdida agregada de la aseguradora por la cobertura $c$ por unidad de tiempo del portafolio
    \item $S = \sum_{c \in C} S^{(c)}$: Pérdida agregada de la aseguradora por unidad de tiempo del portafolio
\end{itemize}

\subsection{Metodología de estimación para el nuevo portafolio}

Todo el análisis se desarrolla con el propósito de modelar las nuevas pólizas, por lo tanto para la frecuencia y la severidad se implementan tratamientos distintos.

\subsubsection{Modelación de frecuencia}

Para la frecuencia, se encuentra por máxima verosimilitud, pruebas de bondad de ajuste y análisis del índice de sobredispersión la mejor distribución entre Poisson o Binomial Negativa para cada $N^{(c)}$ del portafolio histórico. Posteriormente, se trasladan esos parámetros al nuevo portafolio mediante la siguiente metodología:

Se extrae una constante de intensidad para cada cobertura:
\begin{equation*}
\lambda_c = \frac{\text{Promedio de siniestros reportados por la cobertura } c \text{ por unidad de tiempo}}{\text{Exposición promedio del portafolio por la cobertura } c \text{ por unidad de tiempo}}
\end{equation*}

Se supone que esta intensidad por unidad de exposición se mantiene constante entre 2018 y 2019. Así, para el nuevo portafolio se modela:
\begin{equation*}
\mathbb{E}[N^{(c)}] = \lambda_c \times \text{Exposición promedio del nuevo portafolio por la cobertura } c \text{ por unidad de tiempo}
\end{equation*}

Para concluir con el modelo de frecuencia, se supone que entre 2018 y 2019 las distribuciones de las frecuencias son las mismas. Por lo tanto, usando el supuesto anterior y el índice de sobredispersión que también se supone constante, se encuentran los parámetros de las distribuciones de frecuencia del nuevo portafolio.

\subsubsection{Modelación de severidad}

Para la severidad por cobertura $X^{(c)}$, se toma la severidad histórica de cada siniestro, se ajusta con el IPC mensual a enero de 2019 y se utiliza como muestreo para ajustar por máxima verosimilitud y pruebas de bondad de ajuste alguna distribución entre Gamma, Normal, LogNormal, Normal Potenciada o Weibull.

\subsubsection{Pérdida agregada por cobertura}

Para la pérdida agregada por cobertura $S^{(c)}$ se utilizan dos algoritmos alternativos para encontrar su densidad de probabilidad:
\begin{itemize}
    \item Recursión de Panjer
    \item Transformada Rápida de Fourier (FFT)
\end{itemize}

\subsubsection{Pérdida agregada total}

Para la pérdida agregada total $S$ se utiliza la convolución para encontrar su distribución de probabilidad, combinando las distribuciones individuales de cada cobertura.