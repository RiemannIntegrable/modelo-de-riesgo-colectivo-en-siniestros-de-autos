\section{Objetivos}

\subsection{Objetivo General}
Enfrentarse a una base de datos real, aprender a manipularla, extraer información pertinente de los datos, aplicar la teoría vista con relación a los modelos de riesgo colectivos, profundizar en la comprensión de esos conceptos y plantear por sí mismos preguntas coherentes y posibles de resolver.
\subsection{Objetivos Específicos}
\subsubsection{Manipulación y Análisis de Datos}
Desarrollar competencias en la exploración, limpieza y transformación de bases de datos actuariales, identificando variables clave como primas, siniestros y factores de riesgo para generar indicadores de siniestralidad y segmentación del portafolio.
\subsubsection{Aplicación de Modelos de Riesgo Colectivos}
Implementar la teoría de modelos de riesgo colectivos mediante el análisis de frecuencia y severidad de siniestros, calculando métricas fundamentales como:
\begin{equation*}
S = \sum_{i=1}^{N} X_i \quad \text{y} \quad \lambda = \frac{\text{Número de siniestros}}{\text{Exposición}} \text{, Valor esperado de $N$}
\end{equation*}
\subsubsection{Formulación de Preguntas Actuariales Coherentes}
Plantear hipótesis verificables sobre la adecuación tarifaria y comportamiento del riesgo, como evaluar si las diferencias de primas entre segmentos (ej. ciclomotores vs. motocicletas de alto cilindraje) reflejan apropiadamente las diferencias en el riesgo subyacente.