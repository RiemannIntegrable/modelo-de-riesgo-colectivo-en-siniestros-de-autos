\section*{Resumen}

Este estudio presenta el desarrollo y aplicación de un modelo de riesgo colectivo para la evaluación actuarial de un portafolio de seguros de automóviles de 294 pólizas con vigencia mensual desde enero de 2019. La investigación se fundamenta en datos históricos de siniestralidad de 2018 y utiliza técnicas avanzadas de convolución numérica para la determinación de distribuciones de pérdida agregada, primas stop-loss y requerimientos de capital bajo estándares regulatorios de Solvencia II.\\

El marco metodológico implementado aborda las limitaciones estructurales de los datos mediante agregación temporal mensual y homogeneización monetaria vía factores IPC. Se desarrolló un modelo de cuatro coberturas independientes (PPD, PPH, PTH, RC) con modelación diferenciada de frecuencia utilizando distribuciones Poisson y Binomial Negativa, y severidad mediante distribuciones Gamma y Lognormal. La implementación computacional combina algoritmos recursivos de Panjer y Transformada Rápida de Fourier (FFT) con validación cruzada exhaustiva, garantizando robustez numérica y eficiencia en el cálculo de distribuciones agregadas con resolución de \$10,000 COP hasta \$2,000 millones.\\

Los resultados principales incluyen la caracterización de la distribución de pérdida agregada mensual mediante una mixtura gaussiana de cuatro componentes con pérdida esperada de \$12.7 millones COP, la determinación de primas stop-loss individuales para seguros con distribución Gamma ($\alpha=1.874$, $\beta=6.152 \times 10^{-7}$), y el cálculo del requerimiento de capital de solvencia de \$51.6 millones COP mensuales bajo calibración regulatoria al 99.5\%. Este marco actuarial proporciona herramientas cuantitativas robustas para la gestión integral de riesgos, pricing técnico y cumplimiento regulatorio en portafolios de seguros generales de automóviles.