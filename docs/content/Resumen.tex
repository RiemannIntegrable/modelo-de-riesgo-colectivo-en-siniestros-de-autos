\section*{Resumen}

A lo largo de la exploración de datos encontramos que las pólizas nos obligaban a modelar la pérdida agregada $S$ en función de una unidad temporal debido a la ausencia de identificadores únicos en las pólizas. Las nuevas pólizas vigentes que debemos modelar se encuentran vendidas en dos únicos días de enero de 2019 y los datos de siniestros solo se encontraban casi en su totalidad en 2018, por lo que tomamos como input las pólizas que tenían vigencia en 2018 para construir el modelo de pérdida agregada por día. Aplicamos métodos estadísticos básicos para la estimación de las distribuciones de frecuencia y severidad, lo cual deja espacio de mejora aplicando modelos lineales generalizados o algoritmos de machine learning más complejos. Durante la implementación, aplicamos los algoritmos vistos en clase para construir la función de densidad de la pérdida agregada $S$ y en base a este modelo realizamos descripciones actuariales técnicas del nuevo portafolio. 