\section{Limpieza de datos}

Para el modelo propuesto, debemos de realizar una limpieza riguroza de nuestros datos de imput.

\subsection{Limpieza del histórico de pólizas}

La limpieza del histórico de pólizas comprendió varios pasos críticos para asegurar la calidad y consistencia de los datos utilizados en el modelo.

\subsubsection{Estandarización de nombres de columnas}

Se realizó una estandarización de los nombres de columnas del dataset original, transformando la nomenclatura del sistema de la aseguradora a nombres descriptivos en español:

\begin{itemize}
    \item \texttt{FECINICIO} $\rightarrow$ \texttt{Fecha\_inicio}
    \item \texttt{FECFIN} $\rightarrow$ \texttt{Fecha\_fin}
    \item \texttt{VLRPRISUSCR} $\rightarrow$ \texttt{Prima}
    \item \texttt{VLRASEGU} $\rightarrow$ \texttt{Valor\_asegurado}
    \item \texttt{VLRASEGURC} $\rightarrow$ \texttt{Valor\_asegurado\_rc}
    \item Indicadores de cobertura: \texttt{PTD}, \texttt{PPD}, \texttt{PH}, \texttt{PPH}, \texttt{RC} se transformaron a minúsculas
\end{itemize}

\subsubsection{Eliminación de coberturas inconsistentes}

Dado que en la base de datos de siniestros no se registró ningún siniestro con cobertura PTD (Pérdida Total por Daños), se eliminó esta columna del análisis para mantener consistencia entre los datasets de pólizas y siniestros.

\subsubsection{Tratamiento de registros duplicados y valores nulos}

Se eliminaron registros duplicados utilizando la función \texttt{unique()}. Para los valores nulos, se identificaron 263 registros con valores faltantes, representando únicamente el 0.0033\% del total de observaciones. Dada la baja proporción, se aplicó eliminación completa de casos (\textit{listwise deletion}) mediante \texttt{na.omit()}.

\subsubsection{Filtrado temporal}

Siguiendo la estrategia definida en el análisis exploratorio, se seleccionaron únicamente las pólizas que tuvieron exposición durante 2018, aplicando el filtro:

\begin{center}
\texttt{Fecha\_inicio $\leq$ 2018-12-31 AND Fecha\_fin $\geq$ 2018-01-01}
\end{center}

\subsubsection{Depuración por valor de prima}

Se implementó un filtro de calidad basado en el valor de la prima, utilizando como umbral el valor de las tarifas SOAT vigentes en 2016 (COP \$459,500). Este filtro se justifica porque:

\begin{itemize}
    \item Las pólizas con prima inferior al umbral representaron el 18.2\% de los registros
    \item Sin embargo, su contribución al valor total de primas fue únicamente del 1.03\%
    \item Se consideraron probables errores de registro o pólizas con características especiales no representativas del portafolio
\end{itemize}

Tras aplicar este filtro, se conservaron 319,298 pólizas para el análisis.

\subsubsection{Filtrado por duración de vigencia}

Se calculó la duración de cada póliza y se identificó que el 0.30\% de las pólizas tenían vigencia inferior a 60 días. Estas pólizas se eliminaron bajo el supuesto de que correspondían a seguros adquiridos con pagos mensuales que fueron cancelados por impago después de dos meses.

\subsubsection{Generación de datasets de exposición}

Finalmente, se aplicó la función \texttt{polizas\_diarias()} para generar datasets de exposición diaria por cobertura, calculando para cada día de 2018:

\begin{itemize}
    \item Número de pólizas vigentes por cobertura
    \item Exposición acumulada (suma de días-póliza restantes)
    \item Agregación por día del año (variable \texttt{dia})
\end{itemize}

Se generaron también datasets agregados semanalmente y se calcularon las exposiciones totales por cobertura para uso en la calibración de parámetros del nuevo portafolio.

\subsection{Limpieza del histórico de siniestros}

La limpieza del histórico de siniestros siguió un proceso sistemático para preparar los datos de severidad y frecuencia necesarios para el modelo actuarial.

\subsubsection{Estandarización de nombres de columnas}

Se realizó una transformación completa de la nomenclatura técnica del sistema de la aseguradora a nombres descriptivos en español, destacando las siguientes transformaciones principales:

\begin{itemize}
    \item \texttt{FECHASIN} $\rightarrow$ \texttt{fecha}
    \item \texttt{VLRSININCUR} $\rightarrow$ \texttt{severidad}
    \item \texttt{COBERTURA\_FINAL} $\rightarrow$ \texttt{cobertura}
\end{itemize}

\subsubsection{Consolidación de coberturas}

Se unificaron las coberturas de responsabilidad civil para mantener consistencia con el dataset de pólizas:

\begin{itemize}
    \item \texttt{RC BIENES} y \texttt{RC PERS} se consolidaron bajo la etiqueta \texttt{rc}
    \item Se mantuvieron las coberturas \texttt{PTH}, \texttt{PPD} y \texttt{PPH} transformándolas a minúsculas
\end{itemize}

\subsubsection{Filtrado temporal}

Siguiendo la estrategia del análisis exploratorio, se aplicó filtrado temporal:

\begin{itemize}
    \item Se identificaron siniestros de 2017 y 2018
    \item Los registros de 2017 correspondían únicamente al mes de enero
    \item Se decidió conservar únicamente los siniestros de 2018 para mantener consistencia temporal
\end{itemize}

\subsubsection{Tratamiento de valores nulos}

Se identificaron 5,380 valores nulos concentrados exclusivamente en la variable \texttt{Fecha\_pago\_amparo}. Dado que esta variable no es relevante para la modelación de riesgo colectivo, no se requirió tratamiento especial de estos valores faltantes.

\subsubsection{Selección de variable de severidad}

Se seleccionó la variable \texttt{Siniestro\_incurrido} como medida de severidad, ya que representa el valor neto de los siniestros sin descuentos por deducibles u otras alteraciones del valor original del accidente.

\subsubsection{Ajuste por inflación}

Se aplicó ajuste por IPC mensual para llevar todos los valores monetarios a enero de 2019, utilizando la función \texttt{ajuste\_dinero\_ipc()}, permitiendo así una comparación apropiada con las pólizas del nuevo portafolio.

\subsubsection{Depuración por cobertura}

Se implementaron filtros específicos de calidad para cada cobertura, eliminando siniestros con severidades unusualmente bajas que podrían corresponder a errores de registro:

\begin{itemize}
    \item \textbf{PPH (Pérdida Parcial por Hurto)}: Se eliminaron siniestros con severidad < COP \$70,000 (14.88\% de los registros)
    \item \textbf{PTH (Pérdida Total por Hurto)}: Se eliminaron siniestros con severidad < COP \$3,000,000 (21.61\% de los registros)
    \item \textbf{PPD (Pérdida Parcial por Daños)}: Se eliminaron siniestros con severidad < COP \$70,000 (8.92\% de los registros)
    \item \textbf{RC (Responsabilidad Civil)}: Se eliminaron siniestros con severidad < COP \$500,000 (18.61\% de los registros)
\end{itemize}

Estos umbrales se definieron considerando los costos mínimos esperados para cada tipo de cobertura en el mercado colombiano.

\subsubsection{Agregación temporal}

Finalmente, se generaron datasets agregados por día del año para cada cobertura, calculando:

\begin{itemize}
    \item Severidad total por día (suma de severidades de todos los siniestros del día)
    \item Número de siniestros por día
    \item Día del año (variable \texttt{dia} de 1 a 365)
\end{itemize}

\subsection{Limpieza del nuevo portafolio}

La limpieza del nuevo portafolio de pólizas vigentes fue un proceso relativamente directo debido a la menor complejidad y mayor calidad de estos datos.

\subsubsection{Estandarización de nombres de columnas}

Se realizó una transformación de la nomenclatura técnica a nombres descriptivos en español:

\begin{itemize}
    \item \texttt{FECINICIO} $\rightarrow$ \texttt{inicio}
    \item \texttt{FECFIN} $\rightarrow$ \texttt{fin}
    \item \texttt{VLRPRISUSCR} $\rightarrow$ \texttt{prima}
    \item \texttt{VLRASEGU} $\rightarrow$ \texttt{valor\_asegurado}
    \item \texttt{VLRASEGURC} $\rightarrow$ \texttt{valor\_asegurado\_rc}
    \item Indicadores de cobertura: \texttt{PPD}, \texttt{PH}, \texttt{PPH}, \texttt{RC} se transformaron a minúsculas
\end{itemize}

\subsubsection{Eliminación de coberturas inconsistentes}

Siguiendo la estrategia aplicada en los datasets históricos, se eliminó la cobertura PTD para mantener consistencia con los datos de siniestros disponibles.

\subsubsection{Consolidación de valores asegurados}

Se unificaron los valores asegurados generales y de responsabilidad civil en una sola variable:

\begin{center}
\texttt{valor\_asegurado = valor\_asegurado + valor\_asegurado\_rc}
\end{center}

\subsubsection{Tratamiento de registros duplicados}

Se identificaron 6 registros duplicados de un total de 300 pólizas (2\% del dataset). Estos duplicados se eliminaron utilizando la función \texttt{unique()}, resultando en 294 pólizas únicas para el análisis.

\subsubsection{Análisis y corrección de duración de vigencia}

El análisis de las fechas de inicio y fin reveló:

\begin{itemize}
    \item Fechas de inicio concentradas entre el 8 y 9 de enero de 2019
    \item Fechas de fin distribuidas entre enero de 2019 y diciembre de 2020
    \item El 50\% de las pólizas tenían duración de 366 días (año bisiesto)
    \item El 13.27\% de las pólizas presentaban duración de 1 día
\end{itemize}

Las pólizas con duración de 1 día se consideraron anomalías de digitación y se corrigieron asignándoles una duración estándar de 366 días, bajo el supuesto de que correspondían a pólizas anuales mal registradas.

\subsubsection{Análisis de estructura de primas}

El análisis de las primas mostró:

\begin{itemize}
    \item Rango de primas: COP \$0 a COP \$4,212,917
    \item El 25.17\% de las pólizas tenían prima inferior a COP \$100,000
\end{itemize}

A diferencia del portafolio histórico, no se aplicó filtrado por valor mínimo de prima, considerando que las primas bajas podrían corresponder a seguros con pagos mensuales donde solo se había registrado la primera cuota.

\subsubsection{Segmentación por cobertura}

Finalmente, se generaron datasets específicos por cobertura, manteniendo únicamente las variables relevantes para cada tipo:

\begin{itemize}
    \item \textbf{PPD}: 289 pólizas con cobertura de pérdida parcial por daños
    \item \textbf{PTH}: 289 pólizas con cobertura de pérdida total por hurto  
    \item \textbf{PPH}: 289 pólizas con cobertura de pérdida parcial por hurto
    \item \textbf{RC}: 224 pólizas con cobertura de responsabilidad civil
\end{itemize}

Esta segmentación permitió el cálculo posterior de exposiciones específicas por cobertura para la calibración de parámetros del modelo de riesgo colectivo del nuevo portafolio.